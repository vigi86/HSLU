\part{SW 13}
\section{Lernziele (Leitfragen)}
\begin{itemize}
    \item Was ist eine VPN? Wieso ist es «Virtual», wieso ist es «Private»?
    \item Was sind die Vorteile von VPNs im Vergleich zu traditionellen privaten Netzwerken?
    \item Was sind die Hauptarten von VPNs?
    \item Was sind die Hauptarten von Remote Access VPNs?
    \item Was ist IPSec? Was ist seine Verwendung?
    \item Woraus besteht eine IPSec Security Association?
    \item Was ist der Unterschied zwischen Transport und Tunnel Modi in IPSec?
    \item Was ist der Unterschied zwischen «Effectiveness» (Wirksamkeit) und «Efficiency» (Effizienz)?
    \item Was ist der Unterschied zwischen «Fault-tolerance» (Fehlertoleranz) und «Resiliency» (Resilienz)?
    \item Geben Sie Beispiele von Fehlertoleranz in Verbindung mit Netzwerktechnologien
    \item Geben Sie Beispiele von Resilienz in Verbindung mit Netzwerktechnologien
    \item Was ist Skalierbarkeit und wie wird es normalerweise erreicht?
    \item Beschreiben Sie zwei «Load Balancing» (Lastverteilung) Strategien
    \item Beschreiben Strategien um «single-points-of-failure» zu vermeiden
    \item Was sind die Hauptunterschiede zwischen 1G und gegenwärtigen Generationen (4G, 5G) von Mobilnetzwerke?
    \item Was ist die Verbindung zwischen IoT und Mobilnetzwerken (insbesondere 5G)?
    \item Warum werden nicht alle Mobilfunknetze auf den neuesten Standard aufgerüstet? Warum koexistieren verschiedene Mobilfunknetzgenerationen nebeneinander?
    \item Nennen Sie Beispiele für Anwendungsfälle von 5G-Netzen
    \item Nennen Sie Beispiele für die im IoT verwendeten Kommunikationstechnologien
    \item Nennen Sie Beispiele für die im IoT verwendeten Backend-Technologien
    \item Nennen Sie Beispiele für Anwendungsfälle von IoT
\end{itemize}

\section{Antworten}
\subsection*{Was ist eine VPN? Wieso ist es «Virtual», wieso ist es «Private»?}
Virtual Private Network.
\begin{itemize}
    \item End-to-end: private Netzwerkverbindung über öffentliche Netzwerke
    \item Virtual: Informationen werden über öffentliches Netzwerk transportiert
    \item Private: Verkehr ist verschlüsselt um die Vertraulichkeit der Daten während dem Transport über dem öffentlichen Netzwerk zu wahren
\end{itemize}
\subsection*{Was sind die Vorteile von VPNs im Vergleich zu traditionellen privaten Netzwerken?}
\begin{itemize}
    \item Kostensparend: Firmen können Kosten reduzieren und gleichzeitig Bandbreite erhöhen.
    \item Sicherheit: Verschlüsselungs- und Authentifizierungsprotokolle schützen Daten vor unbefugtem Zugriff
    \item Erweiterbarkeit: VPNs ermöglichen Firmen das Internet zu nutzen,
\end{itemize}
\subsection*{Was sind die Hauptarten von VPNs?}
\begin{itemize}
    \item Remote-access: dynamisch erstellt zwischen Client und VPN-Gateway
    \item Site-to-site: VPN-Gateways stellen eine ständige Verbindung her. Clients wissen nicht, dass im Hintergrund eine VPN existiert.
    \item
\end{itemize}
\subsection*{Was sind die Hauptarten von Remote Access VPNs?}
\begin{itemize}
    \item Clientless VPN Connection
    \item Client-based VPN connection
\end{itemize}
\subsection*{Was ist IPSec? Was ist seine Verwendung?}

\subsection*{Woraus besteht eine IPSec Security Association?}
\subsection*{Was ist der Unterschied zwischen Transport und Tunnel Modi in IPSec?}

\subsection*{Was ist der Unterschied zwischen «Effectiveness» (Wirksamkeit) und «Efficiency» (Effizienz)?}
\subsection*{Was ist der Unterschied zwischen «Fault-tolerance» (Fehlertoleranz) und «Resiliency» (Resilienz)?}
\begin{itemize}
    \item Fault-tolerance: der Nutzer spürt von der Ausfallsicherung nichts
    \item Fault-resilience: es gibt eine kleine (merkbare) Periode, bei dem der Nutzer einen Unterbruch feststellen könnte
\end{itemize}
\subsection*{Geben Sie Beispiele von Fehlertoleranz in Verbindung mit Netzwerktechnologien}
Meshed Network
\subsection*{Geben Sie Beispiele von Resilienz in Verbindung mit Netzwerktechnologien}
//TODO Beispiel Virtual Router
\subsection*{Was ist Skalierbarkeit und wie wird es normalerweise erreicht?}

\subsection*{Beschreiben Sie zwei «Load Balancing» (Lastverteilung) Strategien}
Round-Robin und Lastabfrage
\subsection*{Beschreiben Strategien um «single-points-of-failure» zu vermeiden}

\subsection*{Was sind die Hauptunterschiede zwischen 1G und gegenwärtigen Generationen (4G, 5G) von Mobilnetzwerke?}
5G hat viel viel mehr Antennen als 1 G, Verbindung ist um einiges stärker
Grosse Senderaten
\subsection*{Was ist die Verbindung zwischen IoT und Mobilnetzwerken (insbesondere 5G)?}
Chirurg der eine Operation in den USA aus der Schweiz macht
Verbindung MUSS STABIL SEIN
Deswegen braucht es starke Mobilnetze
Autonome verbundene Autos müssen effizient und zuverlässig miteinander kommunizieren könenn

\subsection*{Warum werden nicht alle Mobilfunknetze auf den neuesten Standard aufgerüstet? Warum koexistieren verschiedene Mobilfunknetzgenerationen nebeneinander?}
Als Backups der anderen Netze
Teuer

\subsection*{Nennen Sie Beispiele für Anwendungsfälle von 5G-Netzen}
Augmented reality
Remote Collaboration
Immersive Gaming

\subsection*{Nennen Sie Beispiele für die im IoT verwendeten Kommunikationstechnologien}
Full Duplex
Senden und Daten gleichzeitig empfangen
Master Slave Verbindungen (Half duplex) 


\subsection*{Nennen Sie Beispiele für die im IoT verwendeten Backend-Technologien}
JavaScript
Python
PHP
.NET

\subsection*{Nennen Sie Beispiele für Anwendungsfälle von IoT}
Toilette die Verbunden ist
Lampen
Staubsauger
Autos

