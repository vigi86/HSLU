\part{SW 07 - Transport Layer - Transportschicht}
\section{Lernziele (Leitfragen)}
\begin{itemize}
    \item Was ist der Zweck der Transportschicht?
    \item Was für Protokolle findet man in der Transportschicht?
    \item Was sind die wichtigsten Merkmale des TCP Protokolls?
    \item Was sind die wichtigsten Merkmale des UDP Protokolls?
    \item Wozu werden Ports in der Transportschicht verwendet?
    \item Was ist ein Socket?
    \item Was ist ein \flqq{}Socket Pair\frqq?
    \item Geben Sie Beispiele von Anwendungen die TCP verwenden
    \item Für welche Applikationsarten ist UDP besser geeignet als TCP?
    \item Welches Portintervall verwenden normalerweise bekannte Netzwerkapplikationen und -dienste?
    \item Wie realisiert TCP zuverlässige Verbindungen?
    \item Was ist der Zweck des TCP Handshake?
    \item Wie funktioniert der TCP Handshake?
    \item Wie werden Verbindungen in TCP richtig beendet?
    \item Was ist der Zweck von \flqq{}Selective Acknowledgements\frqq?
\end{itemize}

\section{Antworten}
\subsection*{Was ist der Zweck der Transportschicht?}
\begin{itemize}
    \item Logische Kommunikation zwischen Applikationen, welche auf verschiedenen Hosts laufen
    \item Link zwischen Application Layer und darunterliegenden Layern
    \item Individuelle Kommunikationen verfolgen (jeder Tab im Browser) //TODO pic
    \item Segmentierung der Daten und wieder zusammenfügen
    \item Header Information hinzufügen
    \item Identifizieren, Teilen und verschiedene Konversationen managen
    \item Segmentierung //TODO Folie schauen
\end{itemize}
\subsection*{Was für Protokolle findet man in der Transportschicht?}
\begin{itemize}
    \item TCP - Transmission Control Protocoll
    \begin{itemize}
        \item Zuverlässigkeit - Reliability
        \begin{itemize}
            \item Nummerieren von Datensegmenten
            \item Bestätigen von übertragenen Daten
            \item Erneutes Senden von Daten, wenn Zeit abgelaufen
            \item Reorganisation von Daten, wenn in falscher Reihenfolge empfangen: $1,3,5,4,2 \rightarrow 1,2,3,4,5$
        \end{itemize}
        \item Durchsatzkontrolle - Flow Control
        \begin{itemize}
            \item Effizienteste Rate für Empfänger
        \end{itemize}
    \end{itemize}
    \item UDP - User Datagram Protocol
    \begin{itemize}
        \item //TODO
    \end{itemize}
\end{itemize}
\subsection*{Was sind die wichtigsten Merkmale des TCP Protokolls?}
//TODO
Ergänzung Lorisss
\subsection*{Was sind die wichtigsten Merkmale des UDP Protokolls?}
//TODO
\subsection*{Wozu werden Ports in der Transportschicht verwendet?}
//TODO
\subsection*{Was ist ein Socket?}
Ein Socket ist die Kombination von Source IP Address \& Source Port oder Destination IP Address \& Destination Port
\subsection*{Was ist ein \flqq{}Socket Pair\frqq?}
Unique Identifier für eine Verbindung.
\subsection*{Geben Sie Beispiele von Anwendungen die TCP verwenden}
\begin{itemize}
    \item Mail (POP, IMAP)
    \item Secure Shell (SSH)
    \item FTP
    \item HTTP
\end{itemize}
\subsection*{Für welche Applikationsarten ist UDP besser geeignet als TCP?}
\begin{itemize}
    \item DHCP
    \item DNS
    \item SNMP
    \item TFTP
    \item VoIP
    \item Video Conferencing
\end{itemize}
\subsection*{Welches Portintervall verwenden normalerweise bekannte Netzwerkapplikationen und -dienste?}
\begin{itemize}
    \item Low Ports / Well-known Ports: 0-1023, //TODO
    \item Registered Ports: 1024-49151, //TODO
    \item Private and/or Dynamic Ports: 49152-65535, //TODO
\end{itemize}
\subsection*{Wie realisiert TCP zuverlässige Verbindungen?}
//TODO
\subsection*{Was ist der Zweck des TCP Handshake?}
\begin{itemize}
    \item Wissen, dass Server da ist
    \item Client ist fähig Verbindung herzustellen
    \item Server weiss, dass Client verbinden möchte
    \item Vereinbarung zwischen Geräten über Session Control Parametern und optionalen Eigenschaften
\end{itemize}
\subsection*{Wie funktioniert der TCP Handshake?}
//TODO
\subsection*{Wie werden Verbindungen in TCP richtig beendet?}
//TODO
\subsection*{Was ist der Zweck von \flqq{}Selective Acknowledgements\frqq?}
//TODO