\part{SW 05/06 - Network Layer - Vermittlungsschicht}\label{part:sw0506}
\section{Lernziele (Leitfragen) SW 05}
\begin{itemize}
    \item Was ist der Zweck der Vermittlungsschicht?
    \item Was für Protokolle findet man in der Vermittlungsschicht?
    \item Was sind die wichtigsten Merkmale des IPv4 Protokolls?
    \item Wie lange sind IPv4 Adressen?
    \item Wie sind IPv4 Adressen unterteilt?
    \item Wie findet man die Netzwerkadresse anhand der Hostadresse und der Subnetzmaske?
    \item Was ist die Verbindung zwischen Subnetzmasken und «Slash Notation»?
    \item Was ist der Unterschied zwischen Private und Public IPv4 Adressen?
    \item Wie werden Private IPv4 Adressen verwendet im Internet?
    \item Wieso brauchen wir Private IPv4 Adressen?
    \item Was ist eine Loopbackadresse? Wie wird diese Adresse verwendet?
    \item Was sind «Link-Local» (APIPA) Adressen? Wie und wann werden diese Adressen verwendet?
    \item Wie routet ein Host seine eigenen IPv4 Pakete?
    \item Was ist die Rolle der Default Gateway in dem Routing Prozess?
\end{itemize}

\section{Antworten}
\subsection*{Was ist der Zweck der Vermittlungsschicht?}
\begin{itemize}
    \item *Addressing end devices
    \item Encapsulation
    \begin{itemize}
        \item IP encaplsulates the transport layer segment
        \item IP can use either an IPv4 or IPv6 packet and not impact the layer 4 segment
        \item IP packet will be examined by all layer 3 devices as it traverses the network
        \item The IP addressing does not change from source to destination (except when NAT is used)
    \end{itemize}
    \item *Routing
    \item De-Encapsulation
\end{itemize}

\subsection*{Was für Protokolle findet man in der Vermittlungsschicht?}
\subsection*{Was sind die wichtigsten Merkmale des IPv4 Protokolls?}
Network Layer is connectionless
\begin{itemize}
    \item No connection (establishment): packets are just sent
    \item No control information (synchronizations, acknowledgements, etc.)
    \item The destination will receive the packet... hopefully!
\end{itemize}

Network Layer does best effort
\begin{itemize}
    \item No delivery guarantee
    \item No mechanism to resend data
    \item Does not know if the other device is operational or if it received the packet
\end{itemize}

Network Layer is media independent
\begin{itemize}
    \item IP does not care about the Data Link Layer or the Physical Layer
    \item With one exception: try not to exceed the Data Link Layer Maximum Transfer Unit (MTU)
    \begin{itemize}
        \item MTU must be provided by the Data Link Layer
        \item Undesirable for the Network Layer packet size to exceed the DL Layer MTU
        \item What happens if the IP packet is larger than the DL MTU?
    \end{itemize}
\end{itemize}
\subsection*{Wie lange sind IPv4 Adressen?}
4 bytes = 32 bits
\subsection*{Wie sind IPv4 Adressen unterteilt?}
\begin{itemize}
    \item Network Address
    \item Host Address
    \item Broadcast Address
\end{itemize}
\subsection*{Wie findet man die Netzwerkadresse anhand der Hostadresse und der Subnetzmaske?}
//TODO
\subsection*{Was ist die Verbindung zwischen Subnetzmasken und «Slash Notation»?}
//TODO
\subsection*{Was ist der Unterschied zwischen Private und Public IPv4 Adressen?}\label{sub:private_public_IP}
Auf private IPv4 Adressen kann von aussen nicht direkt zugegriffen werden. Diese sind nach aussen hin unsichtbar.
\subsection*{Wie werden Private IPv4 Adressen verwendet im Internet?}
NAT
\subsection*{Wieso brauchen wir Private IPv4 Adressen?}
Um innerhalb des LANs auf Endgeräte zugreifen zu können.
\subsection*{Was ist eine Loopbackadresse? Wie wird diese Adresse verwendet?}
Die Loopbackadresse zeigt auf den eigenen Host. Diese wird meistens dazu genutzt, um Programme, die als Server dienen können, lokal zu betreiben.
\subsection*{Was sind «Link-Local» (APIPA) Adressen? Wie und wann werden diese Adressen verwendet?}\index{\acrfull{apipa}}\glsadd{APIPA}\label{sub:APIPA}
//TODO
\subsection*{Wie routet ein Host seine eigenen IPv4 Pakete?}
//TODO
\subsection*{Was ist die Rolle der Default Gateway in dem Routing Prozess?}
//TODO

%? //TODO http://www.steves-internet-guide.com/ipv6-guide/