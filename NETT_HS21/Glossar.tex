%! Akronyme und Glossar

% \acrshort{pwm}    -> Ausgabe: PWM
% \acrlong{pwm}     -> Ausgabe: Pulsweitenmodulation
% \acrfull{pwm}     -> Ausgabe: Pulsweitenmodulation (PWM)

% \newacronym{label}{short}{long}
\newacronym{apipa}{APIPA}{Automatic Private IP Addressing}
\newacronym{boyd}{BOYD}{Bring your own Device}
\newacronym{dsl}{DSL}{Digital Subscriber Line}
\newacronym{ietf}{IETF}{Internet Engineering Task Force}
\newacronym{gsm}{GSM}{Global System for Mobile Communication}
\newacronym{iot}{IoT}{Internet of Things}
\newacronym{mac}{MAC}{Media Access Control}
\newacronym{nic}{NIC}{Network Interface Controller/Card}
\newacronym{voip}{VoIP}{Voice over IP}

% \newglossaryentry{label}{name={%<name%>},description={%<description%>}}
\newglossaryentry{APIPA}
{
  name={\acrlong{apipa}},
  description={\acrfull{apipa} ist eine sogenannte Link-Local Address. Es ist eine vom Betriebssystem automatisch zugewiesene IP-Adresse, falls das System auf DHCP eingestellt ist, jedoch nichts vom DHCP offeriert wurde. Dies weil entweder kein DHCP-Server im Netzwerk vorhanden ist oder dieser keine Antwort gibt.\\Der Adressbereich in IPv4 ist 169.254.0.0/16 (196.254.0.0 - 169.254.255.255).}
}
\newglossaryentry{IoT}
{
  name={\acrlong{iot}},
  description={Begriff für Technologien einer globalen Infrastruktur der Informationsgesellschaften, die es ermöglicht, physische und virtuelle Objekte miteinander zu vernetzen und sie durch Informations- und Kommunikationstechniken zusammenarbeiten zu lassen. Vernetzung allerlei Dinge von der Wetterstation zuhause, dem intelligenten Kühlschrank bis hin zum selbstfahrenden Auto.}
}
\newglossaryentry{NIC}
{
  name={\acrlong{nic}},
  description={Ein \acrshort{nic} ist die Netzwerkkarte eines Clients.}
}