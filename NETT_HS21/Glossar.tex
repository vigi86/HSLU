%! Akronyme und Glossar

% \acrshort{pwm}    -> Ausgabe: PWM
% \acrlong{pwm}     -> Ausgabe: Pulsweitenmodulation
% \acrfull{pwm}     -> Ausgabe: Pulsweitenmodulation (PWM)

% \newacronym{label}{short}{long}
\newacronym{apipa}{APIPA}{Automatic Private IP Addressing}
\newacronym{boyd}{BOYD}{Bring your own Device}
\newacronym{dsl}{DSL}{Digital Subscriber Line}
\newacronym{ietf}{IETF}{Internet Engineering Task Force}
\newacronym{gsm}{GSM}{Global System for Mobile Communication}
\newacronym{iot}{IoT}{Internet of Things}
\newacronym{mac}{MAC}{Media Access Control}
\newacronym{nat}{NAT}{Network Address Translation}
\newacronym{nic}{NIC}{Network Interface Controller/Card}
\newacronym{voip}{VoIP}{Voice over IP}

% \newglossaryentry{label}{name={%<name%>},description={%<description%>}}
\newglossaryentry{IoT}
{
  name={\acrlong{iot}},
  description={Begriff für Technologien einer globalen Infrastruktur der Informationsgesellschaften, die es ermöglicht, physische und virtuelle Objekte miteinander zu vernetzen und sie durch Informations- und Kommunikationstechniken zusammenarbeiten zu lassen. Vernetzung allerlei Dinge von der Wetterstation zuhause, dem intelligenten Kühlschrank bis hin zum selbstfahrenden Auto.}
}
\newglossaryentry{NAT}
{
  name={Network Address Translation},
  description={Ein Router übernimmt die \glqqÜbersetzung\grqq{} von privaten IP Adressen in eine öffentliche, damit eine Anfrage ins Internet (z.B. Webseitenaufruf) wieder zum eigenen Netzwerk zurückfindet.}
}
\newglossaryentry{NIC}
{
  name={\acrlong{nic}},
  description={Ein \acrshort{nic} ist die Netzwerkkarte eines Clients.}
}