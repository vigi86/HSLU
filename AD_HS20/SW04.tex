\part{SW 04 - Hashtabellen}
\section{Lernziele}
\begin{itemize}
    \item Sie verstehen wie Hashtabellen funktionieren
    \item Sie kennen verschiedene Implementationsvarianten von Hashtabellen
    \item Sie sind sich der Wichtigkeit guter Hashwerte im Klaren
    \item Sie kennen verschiedene Varianten zur Kollisionsbehandlung bei Hashtabellen
    \item Sie haben eine Vorstellung über den Ablauf und den Aufwand der grundlegenden Operationen auf Hashtabellen
    \item Sie können für die jeweiligen Szenarien geeignete Datenstrukturen auswählen und beurteilen
\end{itemize}

\section{Hashwerte für Datenstrukturen nutzen}
\subsection{Grundlagen}
\subsection{Berechnung}

\section{Hashtabelle - Grundidee}
\subsection{Grundidee}

\section{Kollisionen}
\begin{itemize}
    \item
\end{itemize}
\subsection{Umgang mit Kollisionen}
\subsection{Sondieren}

\section{Operationen}
\subsection{Grundlagen}
\subsection{Einfügen (ohne Kollisionen)}
\subsection{Einfügen (mit Kollisionen)}
\subsection{Suchen}
\subsubsection{Einfache Fälle}
\subsubsection{Enthaltenes Element mit Kollision}
\subsubsection{Nicht Enthaltenes Element mit Kollision}
\subsection{Entfernen - Fall 1}
\subsection{Entfernen - Fall 2}
\subsection{Ununterbrochene Sondierungskette}
\subsection{Entfernen eines Elementes - mit Grabstein}
\subsection{Kollisionsbehandlung mit Sondierungskette}

\section{Hashtabellen mit verketteten Listen}
\begin{itemize}
    \item
\end{itemize}
\subsection{Hashtabellen mit Listen - Vor- und Nachteile}

\section{Wichtige Rahmenbedingungen für Hashtabellen}
\begin{itemize}
    \item
\end{itemize}
\subsection{Empfehlung - Immutable Objects}

\section{Java: Hash-basierende Datenstrukturen}
\subsection{Java Collection Framework - Hash-Datenstrukturen}
\subsection{\texttt{equals()} und \texttt{hashCode()}}

\part{SW 04 - Datenstrukturen: Tipps für die (Java-)Praxis}
\section{Lernziele}
\begin{itemize}
    \item Sie können eine achtsame Auswahl der geeigneten Datenstrukturen treffen
    \item Sie kennen verschiedene Tipps und Hinweise beim Umgang mit Datenstrukturen
    \item Sie können typische Programmierfehler im Zusammenhang mit Datenstrukturen bei Java vermeiden
    \item Sie kennen ausgewählte Hinweise aus dem Buch "`Effective Java"' von Joshua Bloch
    \item Sie kennen alternative Datenstrukturen (Thirdparties) und können diese beurteilen
\end{itemize}

\section{Datenstrukturen und Nebenläufigkeit}
\subsection{"`Veraltete"' Implementationen}
\subsection{Collections sind nicht thread safe implementiert}
\subsection{"`Synchronized"' ist nicht gleich "`Concurrent"'!}

\section{Ergänzende Hinweise zu \texttt{equals()}}
\subsection{Hinweise zur Implementation von \texttt{equals()}}
\subsection{Empfehlung für gute \texttt{equals()}-Methoden}

\section{Ergänzende Hinweise zu \texttt{hashCode()}}
\subsection{Hinweise zur Implementation von \texttt{hashCode()}}
\subsection{Hashwerte von elementaren Datentypen}
\subsection{Empfehlung für gute \texttt{hashCode()}-Methoden}

\section{Minimiere die Mutierbarkeit (Immutable Objects)}
\begin{itemize}
    \item
\end{itemize}
\subsection{Eigenschaften einer unveränderbaren Klasse}
\subsection{Beispiel einer unveränderlichen Klasse - \texttt{Point}}
\subsection{Vorteile von unveränderbaren Objekten}
\subsection{Empfehlung - Immutable Klassen }
\subsection{Immutable bzw. Unmodifiable Collections}

\section{Leere Collections, nicht \texttt{null}}
\subsection{Rückgabe von \texttt{null}-Objects}
\subsection{\texttt{null}-Werte müssen immer explizit behandelt werden}
\subsection{Rückgabe von \texttt{null}-Werten ist fehleranfälliger}
\subsection{Empfehlung - Verwendung von "`empty collections"'}

\section{Generische Datenstrukturen (ohne raw-Types) verwenden}
\subsection{Generische Klassen}
\subsection{Keine raw-Types mehr verwenden}
\subsection{Warum raw-Types schlecht sind}
\subsection{Beispiele}
\subsubsection{Schlechtes Beispiel: Verwendung des raw-Types}
\subsubsection{Gutes Beispiel: Parametrisierbarer Typ (generisch)}
\subsection{Empfehlung zu Generics}

\section{Präferiere Colections vor Arrays}
\subsection{Generische Listen sind besser als Arrays}
\subsection{Kovarianz und Invarianz}
\subsection{Hintergrund: Reify und erasure}
\subsection{Es gibt keine generischen Arrays in Java!}
\subsection{Empfehlung - Collections den Arrays vorziehen}

\section{Thirdparty Datenstrukturen}
\begin{itemize}
    \item
\end{itemize}
\subsection{Beispiele von Thirdparty-Datenstrukturen Libraries}
\subsection{Empfehlungen - Thirdparty Collection Libraries}