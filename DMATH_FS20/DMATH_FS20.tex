\documentclass[12pt,a4paper]{article}
\usepackage{ngerman}
\usepackage[utf8]{inputenc}
\usepackage{sectsty, xcolor}
\usepackage{lastpage}
\usepackage{amsmath, amsfonts, amssymb, enumitem, fancyhdr, graphicx, float, makeidx, textcomp, multicol}
\usepackage[hidelinks]{hyperref}
\usepackage[hang,flushmargin]{footmisc}
\makeindex

\definecolor{dunkelblau}{rgb}{0,0.4,0.6}
\subsectionfont{\color{dunkelblau}}

\title{DMATH FS 2020}
\author{Victor Fernández}
\date{Januar 2020}

\addtolength{\oddsidemargin}{-.875in}
\addtolength{\evensidemargin}{-.875in}
\addtolength{\textwidth}{1.75in}
\addtolength{\topmargin}{-.875in}
\addtolength{\textheight}{1.75in}

% muss nach Änderung der margin kommen!
\pagestyle{fancy}
\fancyhf{} %reset
\fancyhead[L]{HSLU}
\fancyhead[C]{DMATH}
\fancyhead[R]{\thepage/\pageref{LastPage}}
\fancyfoot[L]{}
\fancyfoot[C]{}
\fancyfoot[R]{}
\renewcommand{\headrulewidth}{0.2pt} % Strich in Kopfzeile

\begin{document}

\maketitle
\tableofcontents
\thispagestyle{empty}
\pagebreak

\part{Logik und Beweise}
\section{Logik}
\subsection{Propositionen (Aussagen)}Eine Proposition ist ein Satz, der entweder wahr (Wahrheitswert w) oder falsch (Wahrheitswert f) ist.
\subsection{Negation}Ist $p$ eine Propostion, dann ist die Proposition "`Es ist nicht der Fall, dass p gilt"' die Negation von $p$; man schreibt $\neg p$ und liest "`nicht p"'.

\subsection{Wahrheitstabelle}Die Wahrheitstabelle stellt die Beziehungen zwischen den Wahrheitswerten von Propositionen dar. Sie ist vor allem dann nützlich, wenn Propositionen aus einfachen Propositionen konstruiert werden.\\
\begin{tabular}{|c|c|}
    \hline
        $p$&$\neg p$\\
        \hline
        w&f\\
        f&w\\
    \hline
\end{tabular}

\subsection{Konjunktion - UND-Verknüpfung}Die Propositionen $p\wedge q$ (gelesen "`p und q"') heisst Konjunktion der Propositionen p und q, falls diese genau dann wahr ist, wenn p und q wahr sind; andernfalls ist sie falsch.

\subsection{Disjunktion - ODER-Verknüpfung}Die Propositionen $p\vee q$ (gelesen "`p oder q"') heisst Disjunktion der Propositionen p und q falls diese wahr ist, wenn mindestens eine der Propositionen p oder q wahr ist; andernfalls ist sie falsch.

\subsection{Konjunktion und Disjunktion}UND- und ODER-Verknüpfung\\
\begin{tabular}{|c|c|c|c|}
    \hline
        $p$&$q$&$p\wedge q$&$p\vee q$\\
        \hline
        w&w&w&w\\
        w&f&f&w\\
        f&w&f&w\\
        f&f&f&f\\
    \hline
\end{tabular}

\subsection{XOR-Verknüpfung (eXklusives OR, EXOR)}Die Propositionen $p \oplus q$ (gelesen "`p exor q"') heisst XOR-Verknüpfung der Propositionen p und q, falls diese genau dann wahr ist, wenn genau eine der Propositionen p oder q wahr ist (aber nicht beide gleichzeitig); ansonsten ist sie falsch.\\
\begin{tabular}{|c|c|c|}
    \hline
        $p$&$q$&$p\oplus q$\\
        \hline
        w&w&f\\
        w&f&w\\
        f&w&w\\
        f&f&f\\
    \hline
\end{tabular}

\subsection{Implikationen (Subjunktion)}Die Implikationen $p \rightarrow q$ (gelesen "`p impliziert q"' oder "`falls p, dann q"') ist diejenige Proposition, die genau dann falsch ist, wenn p wahr und q falsch ist; andernfalls ist die Implikation wahr. p heisst auch \textbf{Hypothese} und q \textbf{Konklusion}.\\
\begin{tabular}{|c|c|c|}
    \hline
        $p$&$q$&$p\rightarrow q$\\
        \hline
        w&w&w\\
        w&f&f\\
        f&w&w\\
        f&f&w\\
    \hline
\end{tabular}

\subsection{Bikonditional (Bijunktion)}Das Bikonditional $p \leftrightarrow q$ (gelesen "`p genau dann, wenn q"') ist diejenige Proposition, die wahr ist, wenn p und q dieselben Wahrheitswerte haben und sons falsch.\\
Beispiel: Falls p = "`Sie können den Flug nehmen"' und q = "`Sie kaufen ein Ticket"' zwei Aussagen sind, dann gilt sicher $p \leftrightarrow q$ was lautet: "`Sie können den Flug nehmen, genau dann, wenn Sie ein Ticket kaufen."'

\subsection{Priorität von Logischen Operatoren}Jeder Operator hat eine Priorität die entscheidet, wann der Operator angewandt wird.\\
\begin{tabular}{|c|c|}
    \hline
        Operator&Priorität\\
        \hline
        $\neg$ &1\\
        \hline
        $\wedge$ &2\\
        $\vee$ &2\\
        \hline
        $\rightarrow$ &3\\
        $\leftrightarrow$ &3\\
    \hline
\end{tabular}

\section{Proportionale Äquivalenzen}

\subsection{Tautologie}Eine zusammengesetzte Aussage, die immer wahr (falsch) ist heisst Tautologie (Kontradiktion oder Widerspruch).\\
\begin{tabular}{|c|c|c|c|}
    \hline
        $p$&$\neg q$&$p \vee \neg q$&$p \wedge \neg q$\\
        \hline
        w&f&w&f\\
        f&w&w&f\\
    \hline
\end{tabular}

\subsection{Logische Äquivalenz}Die Aussagen p und q heissen logisch äquivalent, falls $p \leftrightarrow q$ eine Tautologie ist. Man schreibt dann $p \Leftrightarrow q$ (oder auch $p \equiv q$ bzw. $p \sim q$)
\paragraph{TODO Logische Äquivalenzgesetze}

\section{Prädikate und Quantoren}
\subsection{Prädikate}Ein Prädikat ist eine Folge von Wörtern die Variablen enthalten und für jede (erlaubte) Belegung dieser VAriablen zu einer Aussage werden. Man nennt die Aussage $P(x)$ auch den Wert der proportionalen Funktion P für x.

\subsection{Quantoren}
\paragraph{Allquantor - FOR$\forall$LL}Ist P(x) wahr für alle x aus einer bestimmten Universalmenge, dann schreibt man $\forall x P(x)$ und liest: "`für alle x gilt P(x)"'.
\paragraph{Existenzquantor - $\exists$XISTS}Ist P(x) wahr für mindestens ein x aus einer bestimmten Universalmenge, dann schreibt man $\exists x P(x)$ und liest: "`es existiert ein x für welches P(x) wahr ist"'. Falls genau ein Element x existiert, für welches P(x) wahr ist, dann schreibt man $\exists !xP(x)$.
\paragraph{Gebundene Variabeln}Wird ein Quantor auf eine Variable x angewandt, dann nennt man diese Variable gebunden, ansonsten frei.
\begin{itemize}[noitemsep,topsep=0pt,leftmargin=*]
    \item In  $\forall xQ(x,y)$ ist die Variable x gebunden, die Variable y aber frei.
    \item In $\exists x(P(x) \wedge Q(x)) \vee \forall xR(x)$ sind alle Variablen gebunden.
\end{itemize}
\paragraph{TODO Beispiele und verschachtelte Quantoren}

\section{Beweise}
\subsection{Mathematische Beweise in der Wissenschaft}
\begin{itemize}[noitemsep,topsep=0pt,leftmargin=*]
    \item Ein Satz (Theorem) ist eine Aussage, von der man zeigen kann, dass sie wahr ist.
    \item Um zu zeigen, dass ein Satz wahr ist, verwendet man eine Abfolge (Sequenz) von (wahren) Aussagen, die zusammen ein Argument, genannt Beweis ergeben.
    \item Aussagen können Axiome oder Postulate enthalten (=grundlegende Annahmen der mathematischen Strukturen. Diese werden eben angenommen und müssen daher nicht bewiesen werden).
    \item Durch logisches (also gewissen Regelnd gehorchendes) Schliessen werden Folgerungen gemacht, die zusammen den Beweis ergeben.
    \item Ein Lemma oder Hilfssatz ist ein einfacher Satz, der in Beweisen von komplizierten Sätzen verwendet wird.
    \item Ein Korollar ist eine einfache Folgerung eines Satzes.
\end{itemize}

\subsection{Direkte Beweis}Der direkte Beweis wenn $p$ dann $q$, oder die Implikation $p \rightarrow q$ gründet darauf, dass aufgrund der Richtigkeit von $p$ die Richtigkeit von $q$ folgt.\\
\begin{tabular}{|c|c|c|c|}
    \hline
        $p$&$q$&$p \rightarrow q$&$p \wedge (p \rightarrow q)$\\
        \hline
        1&1&1&1\\
        1&0&0&0\\
        0&1&1&0\\
        0&0&1&0\\
    \hline
\end{tabular}
\subsection{Indirekte Beweis}
\paragraph{Beweis durch Kontraposition}Falls ein direkter Beweis schwierig ist, kann man versuchen einen indirekten Beweis, z.B. din Beweis durch Kontraposition zu führen: statt zu zeigen, dass $p \rightarrow q$ gilt, zeigt man, dass die logisch äquivalente Aussage $\neg q \rightarrow \neg p$ gilt.
\paragraph{Beweis durch Kontradiktion (Widerspruch)}Eine weitere Möglichkeit des indirekten Beweises ist der Beweis durch Kontradiktion (Widerspruch). Um die Aussage $p$ zu beweisen, nimmt man an $\neg p$ sei wahr und führt diese Aussage durch logisches Schliessen auf den Widerspruch $q (g=F)$, d.h. $\neg p \rightarrow q$ (also $\neg p \rightarrow F$). Somit muss $\neg p$ falsch und damit $p$ wahr sein.

\section{Mengenlehre}
\subsection{Definition}
Eine Menge ist eine ungeordntete Zusammenfassung wohldefinierter, unterscheidbarer Objekt, genannt Elemente, zu einem Ganzen.\\
Für irgend ein Objekt $x$ gilt dann bezüglich der Menge $A$ entweder $x\in A$ oder dann als $x\notin A$.
\paragraph{Mengen}
%\usepackage{amsmath}
\begin{align*}
&\mathbb{N}=\{1, 2, 3, \dots\}, \mathbb{N}_0=\{0, 1, 2, 3 \dots\} &&\text{Menge der natürlichen Zahlen (mit Null)}\\
&\mathbb{Z}\{\dots,-3,-2,-1,0,1,2,3,\dots\} &&\text{Menge der ganzen Zahlen}\\
&\mathbb{Z}^+=\{1,2,3,\dots\} &&\text{Menge der positiven ganzen Zahlen}\\
&\mathbb{Q}=\begin{Bmatrix} \frac{q}{p}\bigg|p\in \mathbb{Z}\wedge q\in \mathbb{N} \end{Bmatrix} &&\text{Menge der Brüche}\\
&\mathbb{R} &&\text{Menge der Reellen Zahlen}\\
&\mathbb{C} &&\text{Menge der komplexen Zahlen}\\
\end{align*}
Es braucht $\mathbb{R}$, denn die Gleichung $x^2=2$ hat in $\mathbb{Q}$ keine Lösung. Analog braucht es $\mathbb{C}$, denn die Gleichung $x^2=-1$ hat in $\mathbb{R}$ keine Lösung.

\paragraph{Spezielle Mengen}


\end{document}