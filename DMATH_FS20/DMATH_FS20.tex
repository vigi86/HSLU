\documentclass[10pt,a4paper]{article}
\usepackage{ngerman}
\usepackage[utf8]{inputenc}
\usepackage{sectsty, xcolor}
\usepackage{lastpage}
\usepackage{amssymb, enumitem, fancyhdr, graphicx, float, makeidx, textcomp, multicol}
\usepackage[hidelinks]{hyperref}
\usepackage[hang,flushmargin]{footmisc}
\makeindex

\definecolor{dunkelblau}{rgb}{0,0.4,0.6}
\subsectionfont{\color{dunkelblau}}

\title{DMATH FS 2020}
\author{Victor Fernández}
\date{Januar 2020}

\addtolength{\oddsidemargin}{-.875in}
\addtolength{\evensidemargin}{-.875in}
\addtolength{\textwidth}{1.75in}
\addtolength{\topmargin}{-.875in}
\addtolength{\textheight}{1.75in}

% muss nach Änderung der margin kommen!
\pagestyle{fancy}
\fancyhf{} %reset
\fancyhead[L]{HSLU}
\fancyhead[C]{DMATH}
\fancyhead[R]{\thepage/\pageref{LastPage}}
\fancyfoot[L]{}
\fancyfoot[C]{}
\fancyfoot[R]{}
\renewcommand{\headrulewidth}{0.2pt} % Strich in Kopfzeile

\begin{document}

\maketitle
\tableofcontents
\thispagestyle{empty}
\pagebreak

\part{Logik und Beweise}
\section{Logik}
\subsection{Propositionen (Aussagen)}Eine Proposition ist ein Satz, der entweder wahr (Wahrheitswert w) oder falsch (Wahrheitswert f) ist.
\subsection{Negation}Ist $p$ eine Propostion, dann ist die Proposition "`Es ist nicht der Fall, dass p gilt"' die Negation von $p$; man schreibt $\neg p$ und liest "`nicht p"'.

\subsection{Wahrheitstabelle}Die Wahrheitstabelle stellt die Beziehungen zwischen den Wahrheitswerten von Propositionen dar. Sie ist vor allem dann nützlich, wenn Propositionen aus einfachen Propositionen konstruiert werden.\\
\begin{tabular}{|c|c|}
    \hline
        $p$&$\neg p$\\
        \hline
        w&f\\
        f&w\\
    \hline
\end{tabular}

\subsection{Konjunktion - UND-Verknüpfung}Die Propositionen $p\wedge q$ (gelesen "`p und q"') heisst Konjunktion der Propositionen p und q, falls diese genau dann wahr ist, wenn p und q wahr sind; andernfalls ist sie falsch.

\subsection{Disjunktion - ODER-Verknüpfung}Die Propositionen $p\vee q$ (gelesen "`p oder q"') heisst Disjunktion der Propositionen p und q falls diese wahr ist, wenn mindestens eine der Propositionen p oder q wahr ist; andernfalls ist sie falsch.

\subsection{Konjunktion und Disjunktion}UND- und ODER-Verknüpfung\\
\begin{tabular}{|c|c|c|c|}
    \hline
        $p$&$q$&$p\wedge q$&$p\vee q$\\
        \hline
        w&w&w&w\\
        w&f&f&w\\
        f&w&f&w\\
        f&f&f&f\\
    \hline
\end{tabular}

\subsection{XOR-Verknüpfung (eXklusives OR, EXOR)}Die Propositionen $p \oplus q$ (gelesen "`p exor q"') heisst XOR-Verknüpfung der Propositionen p und q, falls diese genau dann wahr ist, wenn genau eine der Propositionen p oder q wahr ist (aber nicht beide gleichzeitig); ansonsten ist sie falsch.\\
\begin{tabular}{|c|c|c|}
    \hline
        $p$&$q$&$p\oplus q$\\
        \hline
        w&w&f\\
        w&f&w\\
        f&w&w\\
        f&f&f\\
    \hline
\end{tabular}

\subsection{Implikationen (Subjunktion)}Die Implikationen $p \rightarrow q$ (gelesen "`p impliziert q"' oder "`falls p, dann q"') ist diejenige Proposition, die genau dann falsch ist, wenn p wahr und q falsch ist; andernfalls ist die Implikation wahr. p heisst auch \textbf{Hypothese} und q \textbf{Konklusion}.\\
\begin{tabular}{|c|c|c|}
    \hline
        $p$&$q$&$p\rightarrow q$\\
        \hline
        w&w&w\\
        w&f&f\\
        f&w&w\\
        f&f&w\\
    \hline
\end{tabular}

\subsection{Bikonditional (Bijunktion)}Das Bikonditional $p \leftrightarrow q$ (gelesen "`p genau dann, wenn q"') ist diejenige Proposition, die wahr ist, wenn p und q dieselben Wahrheitswerte haben und sons falsch.\\
Beispiel: Falls p = "`Sie können den Flug nehmen"' und q = "`Sie kaufen ein Ticket"' zwei Aussagen sind, dann gilt sicher $p \leftrightarrow q$ was lautet: "`Sie können den Flug nehmen, genau dann, wenn Sie ein Ticket kaufen."'

\subsection{Priorität von Logischen Operatoren}Jeder Operator hat eine Priorität die entscheidet, wann der Operator angewandt wird.\\
\begin{tabular}{|c|c|}
    \hline
        Operator&Priorität\\
        \hline
        $\neg$ &1\\
        \hline
        $\wedge$ &2\\
        $\vee$ &2\\
        \hline
        $\rightarrow$ &3\\
        $\leftrightarrow$ &3\\
    \hline
\end{tabular}

\section{Proportionale Äquivalenzen}

\subsection{Tautologie}Eine zusammengesetzte Aussage, die immer wahr (falsch) ist heisst Tautologie (Kontradiktion oder Widerspruch).\\
\begin{tabular}{|c|c|c|c|}
    \hline
        $p$&$\neg q$&$p \vee \neg q$&$p \wedge \neg q$\\
        \hline
        w&f&w&f\\
        f&w&w&f\\
    \hline
\end{tabular}

\subsection{Logische Äquivalenz}Die Aussagen p und q heissen logisch äquivalent, falls $p \leftrightarrow q$ eine Tautologie ist. Man schreibt dann $p \Leftrightarrow q$ (oder auch $p \equiv q$ bzw. $p \sim q$)
\paragraph{TODO Logische Äquivalenzgesetze}

\end{document}