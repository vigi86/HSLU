\part{SW 01 - Einführung}\label{part:sw01}
\section{Erklären Sie ein ERP-System.}\index{ERP-System}
Ein \textbf{Enterprise Resource Planning System} ist eine integrierte Verwaltung von Hauptgeschäftsprozessen, welche oft in Echtzeit agiert. Es ist eine Reihe integrierter Anwendungen, mit denen ein Unternehmen Daten aus vielen Geschäftsaktivitäten sammeln, speichern, verwalten und interpretieren kann.\\
ERP-Systeme haben normalerweise folgende Merkmale:
\begin{itemize}
    \item ist ein integriertes System
    \item arbeitet in (oder nahe) Echtzeit
    \item gemeinsame Datenbank, die alle Anwendungen unterstützt
    \item einheitliches Erscheinungsbild über alle Module hinweg
\end{itemize}

\section{Nennen Sie einen entscheidenden Vorteil der Datenintegration.}\index{Integration!Daten}
Eine integrierte Datenstruktur oder Datenintegration ist das Vorhandensein bereinigter Daten. Das heisst, es existieren keine doppelten Daten, da alle Module auf eine gemeinsame Datenbank zugreift.

\section{Welche Möglichkeiten der organisatorischen Integration von Informationssystemen kennen Sie?}\index{Integration!Organisatorisch}
Die organisatorische Datenintegration bezeichnet die Zusammenfassung von Menschen, Aufgaben und Stellen zu einer Organisationseinheit. Somit können wir im Allgemeinen von Stammdaten sprechen.
\paragraph{Betriebswirtschaftliche Stammdaten} Kunden, Lieferanten, Dienstleistungen, Material, Stücklisten, Arbeitspläne
\paragraph{Technische Stammdaten} Chemische Zusammensetzungen, Rezepturen, Durchmesser, Oberflächengüte, Härte etc.
\paragraph{Temporäre Vormerkdaten} eröffnete Angebote, offene Debitoren, offene Kreditoren, Teillieferungen etc.
\paragraph{Transfer Daten} werden von einem Programm generiert und vom anderen Programm benötigt. Beispiele: Die Rechnungen eines Tages werden von der Buchhaltung weiterverarbeitet; Die Lohnbuchhaltung stellt die Daten in einem Transferspeicher für die Hauptbuchhaltung bereit etc.
\paragraph{Archiv Daten} Bsp: Auftragseingänge der letzten 36 Monate, Messwerte aus der Qualitätskontrolle, Reparaturen bei Betriebsmitteln etc.

\section{Welche Möglichkeiten der technischen Integration von Informationssystemen kennen Sie?}\index{Integration!Technisch}
Grundsätzlich gibt es drei Arten der technischen Integration.
\begin{itemize}
    \item Verteilt, keine Integration $\rightarrow$ Jede Applikation hat eigene Daten(banken)
    \item Zentralisiert, integriert $\rightarrow$ Jede Applikation greift auf dieselben Daten(bank)
    \item Verteilt, integriert $\rightarrow$ Jede Applikation hat eigene Daten(banken), jedoch sind sie über Schnittstellen verbunden und können gegebenenfalls so Daten austauschen. Bsp: Eine Zeiterfassungssoftware exportiert von Mitarbeitern die Präsenzzeit zum Lohnbuchhaltungsprogramm, welches anhand der geleisteten Stunden den Lohn berechnet.
\end{itemize}

\section{Nennen Sie drei Vorteile einer SW-Single Solution.}\index{Software!Single Solution}
Kostengünstig, keine Schulung und Support nötig (Kosten), einfache Handhabung.

\section{Warum kaufen Unternehmen teure ERP-Systeme? Nennen Sie die Hauptgründe.}\index{ERP-System!Kaufentscheid}
Ein ERP-System bietet die Möglichkeit einer vertikalen, horizontalen oder lateralen Integration, da sie branchenneutral ist. Sprich, ein Unternehmen, welches eine Rückwärtsintegration in Richtung Produktion und Rohstoffbeschaffung macht, kann in diesen Fertigungsbetrieben ein branchenunabhängiges ERP-System verwenden, welches sie im aktuellen Unternehmen auch nutzt. Bei einer Vorwärtsintegration kann beispielsweise der Vertrieb der Ware auch in das ERP-System integriert werden. Bei einer horizontalen integration kommen Unternehmen in der gleichen Verarbeitungs- und Handelsstufe hinzu, beispielsweise als neue Mandanten. Und auch bei einer lateralen Integration bei einer Diversifikationsstrategie kann ein Betrieb einer fremden Verarbeitung- und Handelsstufe (z.B. Forstbetrieb und Spritzgussfabrik) zentralisiert verwaltet werden. Branchenlösungen bauen auf den neutralen Systemen auf und bieten zusätzliche branchenspezifische Funktionen.

\section{Warum werden SW-Einzellösungen durch Standard-SW ersetzt?}\index{Software!Einzellösung}\index{Software!Standard}
Durch das Customizing wird eine Standard-Software auf die individuellen Bedürfnisse und Abläufe des anwendenden Unternehmens zugeschnitten. Der Vorteil integrierten Daten diverser Betriebsbereiche und individueller Funktionen wird hier sichtbar.

\section{Was bedeutet Echtzeit?}\index{Echtzeit}
Echtzeit bedeutet, dass Mutationen der Daten sofort für alle Module zur Verfügung stehen. Bsp: Bestätigt ein Chauffeur die Lieferung der Ware, kann das System beispielsweise ein Bestellvorgang abschliessen.