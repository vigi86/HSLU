\documentclass[10pt,a4paper]{article}
\usepackage{ngerman}
\usepackage[utf8]{inputenc}
\usepackage{sectsty, xcolor}
\usepackage{lastpage}
\usepackage{amsmath, amsfonts, amssymb, enumitem, fancyhdr, graphicx, float, makeidx, textcomp, multicol}
\usepackage[hidelinks]{hyperref}
\usepackage[hang,flushmargin]{footmisc}
\makeindex

\definecolor{dunkelblau}{rgb}{0,0.4,0.6}
\subsectionfont{\color{dunkelblau}}

\title{AD FS 2020}
\author{Victor Fernández}
\date{Januar 2020}

\addtolength{\oddsidemargin}{-.875in}
\addtolength{\evensidemargin}{-.875in}
\addtolength{\textwidth}{1.75in}
\addtolength{\topmargin}{-.875in}
\addtolength{\textheight}{1.75in}

% muss nach Änderung der margin kommen!
\pagestyle{fancy}
\fancyhf{} %reset
\fancyhead[L]{HSLU}
\fancyhead[C]{AD}
\fancyhead[R]{\thepage/\pageref{LastPage}}
\fancyfoot[L]{}
\fancyfoot[C]{}
\fancyfoot[R]{}
\renewcommand{\headrulewidth}{0.2pt} % Strich in Kopfzeile

\begin{document}

\maketitle
\tableofcontents
\thispagestyle{empty}
\pagebreak

\part{Funktionen}
\section{Funktionen}
\paragraph{Definition}Sind $y$ und $x$ veränderliche Grössen, so nennen wir $y$ eine \textbf{Funktion von $x$}, falls $y$ genau einen Wert annimmt, sobald $x$ einen Wert annimmt.
\paragraph{Abbildung}Eine Abbildung von einer Menge $A$ in eine Menge $B$ ist etwas, das jedem Element aus der Menge $A$ ein Element aus $B$ zuordnet.\\
Ist $a \in A$, so bezeichnet man mit $f(a)$ jenes Element aus $B$, welches $a$ durch $f$ zugeordnet wird.

\paragraph{Lineare Funktion}Sei $y$ eine Funktion von $x$, so nennen wir $y$ eine lineare Funktion von $x$, wenn $\Delta y$ direkt proportional zu $\Delta x$ ist.
\begin{align}
    y=m\cdot x + n
\end{align}
\paragraph{Punkt-Steigungs-Formel in einer linearen Funktion}
\begin{align}
    y = y_1+\frac{\Delta y}{\Delta x}\cdot(x-x_1)
\end{align}


\paragraph{Exponentielle Funktion}$y$ ist eine exponentielle Funktion von $x$, falls der Wert von $y$ immer um den gleichen Prozentsatz zunimmt. Jedes Mal wenn der Wert von $x$ um eine Einheit $c$ zunimmt.
\subparagraph{Alternativ}$y$ vervielfacht sich mit dem gleichen Faktor $a$, jedes Mal wenn $x$ um die Konstante $c$ zunimmt.
\begin{align}
    y = y_0 \cdot a^\frac{x}{c}
\end{align}
Normalform (k$>$0 Wachstum; k$<$0 Zerfall):
\begin{align}
    y = y_0 \cdot e^{k\cdot x}
\end{align}
Kontinuierliche Wachstumsrate, Beispiel:
\begin{align}
    Q=250\frac{mg}{cm^3}\cdot 0.6\frac{t}{h}\\
    =250\frac{mg}{cm^3}\cdot(e^{\ln{0.6}})^{\frac{t}{h}}\\
    =250\frac{mg}{cm^3}\cdot e^{\frac{\ln{0.6}}{h}\cdot t}
\end{align}

\paragraph{Basiswechsel}Jede Exponentialfunktion mit Wachstumsfaktor $a$ pro $c$ lässt sich zu jeder beliebigen anderen Basis $b>0,b\neq 1$ darstellen, unter geeigneter Anpassung der Einheit $c$. Also:
\begin{align}
    P=P_0a^{\frac{t}{c}}=P_0b^{\frac{t}{d}}
\end{align}
wobei
\begin{align}
    d=\frac{c}{\log_{b}a}
\end{align}

\end{document}